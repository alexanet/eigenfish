% Generated by Sphinx.
\def\sphinxdocclass{report}
\newif\ifsphinxKeepOldNames \sphinxKeepOldNamestrue
\documentclass[letterpaper,10pt,english]{sphinxmanual}
\usepackage{iftex}

\ifPDFTeX
  \usepackage[utf8]{inputenc}
\fi
\ifdefined\DeclareUnicodeCharacter
  \DeclareUnicodeCharacter{00A0}{\nobreakspace}
\fi
\usepackage{cmap}
\usepackage[T1]{fontenc}
\usepackage{amsmath,amssymb,amstext}
\usepackage{babel}
\usepackage{times}
\usepackage[Sonny]{fncychap}
\usepackage{longtable}
\usepackage{sphinx}
\usepackage{multirow}
\usepackage{eqparbox}


\addto\captionsenglish{\renewcommand{\figurename}{Fig.\@ }}
\addto\captionsenglish{\renewcommand{\tablename}{Table }}
\SetupFloatingEnvironment{literal-block}{name=Listing }

\addto\extrasenglish{\def\pageautorefname{page}}

\setcounter{tocdepth}{3}


\title{Eigenfish Documentation}
\date{Oct 16, 2016}
\release{1.0}
\author{Seth Pendergrass, Zhe Bai and Steven Brunton}
\newcommand{\sphinxlogo}{}
\renewcommand{\releasename}{Release}
\makeindex

\makeatletter
\def\PYG@reset{\let\PYG@it=\relax \let\PYG@bf=\relax%
    \let\PYG@ul=\relax \let\PYG@tc=\relax%
    \let\PYG@bc=\relax \let\PYG@ff=\relax}
\def\PYG@tok#1{\csname PYG@tok@#1\endcsname}
\def\PYG@toks#1+{\ifx\relax#1\empty\else%
    \PYG@tok{#1}\expandafter\PYG@toks\fi}
\def\PYG@do#1{\PYG@bc{\PYG@tc{\PYG@ul{%
    \PYG@it{\PYG@bf{\PYG@ff{#1}}}}}}}
\def\PYG#1#2{\PYG@reset\PYG@toks#1+\relax+\PYG@do{#2}}

\expandafter\def\csname PYG@tok@no\endcsname{\def\PYG@tc##1{\textcolor[rgb]{0.38,0.68,0.84}{##1}}}
\expandafter\def\csname PYG@tok@se\endcsname{\let\PYG@bf=\textbf\def\PYG@tc##1{\textcolor[rgb]{0.25,0.44,0.63}{##1}}}
\expandafter\def\csname PYG@tok@kt\endcsname{\def\PYG@tc##1{\textcolor[rgb]{0.56,0.13,0.00}{##1}}}
\expandafter\def\csname PYG@tok@nv\endcsname{\def\PYG@tc##1{\textcolor[rgb]{0.73,0.38,0.84}{##1}}}
\expandafter\def\csname PYG@tok@k\endcsname{\let\PYG@bf=\textbf\def\PYG@tc##1{\textcolor[rgb]{0.00,0.44,0.13}{##1}}}
\expandafter\def\csname PYG@tok@nc\endcsname{\let\PYG@bf=\textbf\def\PYG@tc##1{\textcolor[rgb]{0.05,0.52,0.71}{##1}}}
\expandafter\def\csname PYG@tok@sh\endcsname{\def\PYG@tc##1{\textcolor[rgb]{0.25,0.44,0.63}{##1}}}
\expandafter\def\csname PYG@tok@w\endcsname{\def\PYG@tc##1{\textcolor[rgb]{0.73,0.73,0.73}{##1}}}
\expandafter\def\csname PYG@tok@na\endcsname{\def\PYG@tc##1{\textcolor[rgb]{0.25,0.44,0.63}{##1}}}
\expandafter\def\csname PYG@tok@mb\endcsname{\def\PYG@tc##1{\textcolor[rgb]{0.13,0.50,0.31}{##1}}}
\expandafter\def\csname PYG@tok@gp\endcsname{\let\PYG@bf=\textbf\def\PYG@tc##1{\textcolor[rgb]{0.78,0.36,0.04}{##1}}}
\expandafter\def\csname PYG@tok@sb\endcsname{\def\PYG@tc##1{\textcolor[rgb]{0.25,0.44,0.63}{##1}}}
\expandafter\def\csname PYG@tok@nf\endcsname{\def\PYG@tc##1{\textcolor[rgb]{0.02,0.16,0.49}{##1}}}
\expandafter\def\csname PYG@tok@sc\endcsname{\def\PYG@tc##1{\textcolor[rgb]{0.25,0.44,0.63}{##1}}}
\expandafter\def\csname PYG@tok@gi\endcsname{\def\PYG@tc##1{\textcolor[rgb]{0.00,0.63,0.00}{##1}}}
\expandafter\def\csname PYG@tok@sd\endcsname{\let\PYG@it=\textit\def\PYG@tc##1{\textcolor[rgb]{0.25,0.44,0.63}{##1}}}
\expandafter\def\csname PYG@tok@cm\endcsname{\let\PYG@it=\textit\def\PYG@tc##1{\textcolor[rgb]{0.25,0.50,0.56}{##1}}}
\expandafter\def\csname PYG@tok@sx\endcsname{\def\PYG@tc##1{\textcolor[rgb]{0.78,0.36,0.04}{##1}}}
\expandafter\def\csname PYG@tok@cpf\endcsname{\let\PYG@it=\textit\def\PYG@tc##1{\textcolor[rgb]{0.25,0.50,0.56}{##1}}}
\expandafter\def\csname PYG@tok@si\endcsname{\let\PYG@it=\textit\def\PYG@tc##1{\textcolor[rgb]{0.44,0.63,0.82}{##1}}}
\expandafter\def\csname PYG@tok@nn\endcsname{\let\PYG@bf=\textbf\def\PYG@tc##1{\textcolor[rgb]{0.05,0.52,0.71}{##1}}}
\expandafter\def\csname PYG@tok@s2\endcsname{\def\PYG@tc##1{\textcolor[rgb]{0.25,0.44,0.63}{##1}}}
\expandafter\def\csname PYG@tok@mf\endcsname{\def\PYG@tc##1{\textcolor[rgb]{0.13,0.50,0.31}{##1}}}
\expandafter\def\csname PYG@tok@o\endcsname{\def\PYG@tc##1{\textcolor[rgb]{0.40,0.40,0.40}{##1}}}
\expandafter\def\csname PYG@tok@vi\endcsname{\def\PYG@tc##1{\textcolor[rgb]{0.73,0.38,0.84}{##1}}}
\expandafter\def\csname PYG@tok@bp\endcsname{\def\PYG@tc##1{\textcolor[rgb]{0.00,0.44,0.13}{##1}}}
\expandafter\def\csname PYG@tok@ss\endcsname{\def\PYG@tc##1{\textcolor[rgb]{0.32,0.47,0.09}{##1}}}
\expandafter\def\csname PYG@tok@cp\endcsname{\def\PYG@tc##1{\textcolor[rgb]{0.00,0.44,0.13}{##1}}}
\expandafter\def\csname PYG@tok@ch\endcsname{\let\PYG@it=\textit\def\PYG@tc##1{\textcolor[rgb]{0.25,0.50,0.56}{##1}}}
\expandafter\def\csname PYG@tok@sr\endcsname{\def\PYG@tc##1{\textcolor[rgb]{0.14,0.33,0.53}{##1}}}
\expandafter\def\csname PYG@tok@gd\endcsname{\def\PYG@tc##1{\textcolor[rgb]{0.63,0.00,0.00}{##1}}}
\expandafter\def\csname PYG@tok@ow\endcsname{\let\PYG@bf=\textbf\def\PYG@tc##1{\textcolor[rgb]{0.00,0.44,0.13}{##1}}}
\expandafter\def\csname PYG@tok@c1\endcsname{\let\PYG@it=\textit\def\PYG@tc##1{\textcolor[rgb]{0.25,0.50,0.56}{##1}}}
\expandafter\def\csname PYG@tok@kn\endcsname{\let\PYG@bf=\textbf\def\PYG@tc##1{\textcolor[rgb]{0.00,0.44,0.13}{##1}}}
\expandafter\def\csname PYG@tok@kr\endcsname{\let\PYG@bf=\textbf\def\PYG@tc##1{\textcolor[rgb]{0.00,0.44,0.13}{##1}}}
\expandafter\def\csname PYG@tok@s1\endcsname{\def\PYG@tc##1{\textcolor[rgb]{0.25,0.44,0.63}{##1}}}
\expandafter\def\csname PYG@tok@cs\endcsname{\def\PYG@tc##1{\textcolor[rgb]{0.25,0.50,0.56}{##1}}\def\PYG@bc##1{\setlength{\fboxsep}{0pt}\colorbox[rgb]{1.00,0.94,0.94}{\strut ##1}}}
\expandafter\def\csname PYG@tok@nl\endcsname{\let\PYG@bf=\textbf\def\PYG@tc##1{\textcolor[rgb]{0.00,0.13,0.44}{##1}}}
\expandafter\def\csname PYG@tok@nd\endcsname{\let\PYG@bf=\textbf\def\PYG@tc##1{\textcolor[rgb]{0.33,0.33,0.33}{##1}}}
\expandafter\def\csname PYG@tok@s\endcsname{\def\PYG@tc##1{\textcolor[rgb]{0.25,0.44,0.63}{##1}}}
\expandafter\def\csname PYG@tok@nb\endcsname{\def\PYG@tc##1{\textcolor[rgb]{0.00,0.44,0.13}{##1}}}
\expandafter\def\csname PYG@tok@ge\endcsname{\let\PYG@it=\textit}
\expandafter\def\csname PYG@tok@kd\endcsname{\let\PYG@bf=\textbf\def\PYG@tc##1{\textcolor[rgb]{0.00,0.44,0.13}{##1}}}
\expandafter\def\csname PYG@tok@gh\endcsname{\let\PYG@bf=\textbf\def\PYG@tc##1{\textcolor[rgb]{0.00,0.00,0.50}{##1}}}
\expandafter\def\csname PYG@tok@nt\endcsname{\let\PYG@bf=\textbf\def\PYG@tc##1{\textcolor[rgb]{0.02,0.16,0.45}{##1}}}
\expandafter\def\csname PYG@tok@gs\endcsname{\let\PYG@bf=\textbf}
\expandafter\def\csname PYG@tok@c\endcsname{\let\PYG@it=\textit\def\PYG@tc##1{\textcolor[rgb]{0.25,0.50,0.56}{##1}}}
\expandafter\def\csname PYG@tok@vg\endcsname{\def\PYG@tc##1{\textcolor[rgb]{0.73,0.38,0.84}{##1}}}
\expandafter\def\csname PYG@tok@ni\endcsname{\let\PYG@bf=\textbf\def\PYG@tc##1{\textcolor[rgb]{0.84,0.33,0.22}{##1}}}
\expandafter\def\csname PYG@tok@vc\endcsname{\def\PYG@tc##1{\textcolor[rgb]{0.73,0.38,0.84}{##1}}}
\expandafter\def\csname PYG@tok@go\endcsname{\def\PYG@tc##1{\textcolor[rgb]{0.20,0.20,0.20}{##1}}}
\expandafter\def\csname PYG@tok@mi\endcsname{\def\PYG@tc##1{\textcolor[rgb]{0.13,0.50,0.31}{##1}}}
\expandafter\def\csname PYG@tok@m\endcsname{\def\PYG@tc##1{\textcolor[rgb]{0.13,0.50,0.31}{##1}}}
\expandafter\def\csname PYG@tok@mh\endcsname{\def\PYG@tc##1{\textcolor[rgb]{0.13,0.50,0.31}{##1}}}
\expandafter\def\csname PYG@tok@ne\endcsname{\def\PYG@tc##1{\textcolor[rgb]{0.00,0.44,0.13}{##1}}}
\expandafter\def\csname PYG@tok@gt\endcsname{\def\PYG@tc##1{\textcolor[rgb]{0.00,0.27,0.87}{##1}}}
\expandafter\def\csname PYG@tok@il\endcsname{\def\PYG@tc##1{\textcolor[rgb]{0.13,0.50,0.31}{##1}}}
\expandafter\def\csname PYG@tok@kc\endcsname{\let\PYG@bf=\textbf\def\PYG@tc##1{\textcolor[rgb]{0.00,0.44,0.13}{##1}}}
\expandafter\def\csname PYG@tok@mo\endcsname{\def\PYG@tc##1{\textcolor[rgb]{0.13,0.50,0.31}{##1}}}
\expandafter\def\csname PYG@tok@gu\endcsname{\let\PYG@bf=\textbf\def\PYG@tc##1{\textcolor[rgb]{0.50,0.00,0.50}{##1}}}
\expandafter\def\csname PYG@tok@err\endcsname{\def\PYG@bc##1{\setlength{\fboxsep}{0pt}\fcolorbox[rgb]{1.00,0.00,0.00}{1,1,1}{\strut ##1}}}
\expandafter\def\csname PYG@tok@kp\endcsname{\def\PYG@tc##1{\textcolor[rgb]{0.00,0.44,0.13}{##1}}}
\expandafter\def\csname PYG@tok@gr\endcsname{\def\PYG@tc##1{\textcolor[rgb]{1.00,0.00,0.00}{##1}}}

\def\PYGZbs{\char`\\}
\def\PYGZus{\char`\_}
\def\PYGZob{\char`\{}
\def\PYGZcb{\char`\}}
\def\PYGZca{\char`\^}
\def\PYGZam{\char`\&}
\def\PYGZlt{\char`\<}
\def\PYGZgt{\char`\>}
\def\PYGZsh{\char`\#}
\def\PYGZpc{\char`\%}
\def\PYGZdl{\char`\$}
\def\PYGZhy{\char`\-}
\def\PYGZsq{\char`\'}
\def\PYGZdq{\char`\"}
\def\PYGZti{\char`\~}
% for compatibility with earlier versions
\def\PYGZat{@}
\def\PYGZlb{[}
\def\PYGZrb{]}
\makeatother

\renewcommand\PYGZsq{\textquotesingle}

\begin{document}

\maketitle
\tableofcontents
\phantomsection\label{index::doc}


Contents:


\chapter{eigenfish package}
\label{eigenfish:eigenfish-package}\label{eigenfish::doc}\label{eigenfish:welcome-to-eigenfish-s-documentation}

\section{Subpackages}
\label{eigenfish:subpackages}

\subsection{eigenfish.classify package}
\label{eigenfish.classify::doc}\label{eigenfish.classify:eigenfish-classify-package}

\subsubsection{Submodules}
\label{eigenfish.classify:submodules}

\subsubsection{eigenfish.classify.classify module}
\label{eigenfish.classify:module-eigenfish.classify.classify}\label{eigenfish.classify:eigenfish-classify-classify-module}\index{eigenfish.classify.classify (module)}\index{Classifier (class in eigenfish.classify.classify)}

\begin{fulllineitems}
\phantomsection\label{eigenfish.classify:eigenfish.classify.classify.Classifier}\pysigline{\sphinxstrong{class }\sphinxcode{eigenfish.classify.classify.}\sphinxbfcode{Classifier}}
Bases: \sphinxcode{object}
\index{classify() (eigenfish.classify.classify.Classifier method)}

\begin{fulllineitems}
\phantomsection\label{eigenfish.classify:eigenfish.classify.classify.Classifier.classify}\pysiglinewithargsret{\sphinxbfcode{classify}}{\emph{data}}{}
Classifies data based on current model.
\begin{quote}\begin{description}
\item[{Parameters}] \leavevmode
\textbf{\texttt{data}} -- Matrix with each column a different sample.

\item[{Returns}] \leavevmode
List of predictions, where return{[}i{]} describes data{[}:, i{]}.

\end{description}\end{quote}

\end{fulllineitems}

\index{cross\_validate() (eigenfish.classify.classify.Classifier method)}

\begin{fulllineitems}
\phantomsection\label{eigenfish.classify:eigenfish.classify.classify.Classifier.cross_validate}\pysiglinewithargsret{\sphinxbfcode{cross\_validate}}{\emph{data}, \emph{labels}}{}
Cross-validates trained model against data with labels.
\begin{quote}\begin{description}
\item[{Parameters}] \leavevmode\begin{itemize}
\item {} 
\textbf{\texttt{data}} -- Matrix with each column a different sample.

\item {} 
\textbf{\texttt{labels}} -- List of labels, each corresponding to a column of data.

\end{itemize}

\item[{Returns}] \leavevmode
Percent labels the same.

\end{description}\end{quote}

\end{fulllineitems}

\index{load() (eigenfish.classify.classify.Classifier method)}

\begin{fulllineitems}
\phantomsection\label{eigenfish.classify:eigenfish.classify.classify.Classifier.load}\pysiglinewithargsret{\sphinxbfcode{load}}{\emph{filename}}{}
Loads trained model from file, overwriting current model. Do not use on
training files you did not create.
\begin{quote}\begin{description}
\item[{Parameters}] \leavevmode
\textbf{\texttt{filename}} -- Name of file to load.

\end{description}\end{quote}

\end{fulllineitems}

\index{save() (eigenfish.classify.classify.Classifier method)}

\begin{fulllineitems}
\phantomsection\label{eigenfish.classify:eigenfish.classify.classify.Classifier.save}\pysiglinewithargsret{\sphinxbfcode{save}}{\emph{filename}}{}
Saves trained model to filename.
\begin{quote}\begin{description}
\item[{Parameters}] \leavevmode
\textbf{\texttt{filename}} -- Name of file to save model as.

\end{description}\end{quote}

\end{fulllineitems}

\index{train() (eigenfish.classify.classify.Classifier method)}

\begin{fulllineitems}
\phantomsection\label{eigenfish.classify:eigenfish.classify.classify.Classifier.train}\pysiglinewithargsret{\sphinxbfcode{train}}{\emph{data}, \emph{labels}}{}
Trains current classifier with matrix data and labels, where labels{[}i{]}
describes data{[}:, i{]}.
\begin{quote}\begin{description}
\item[{Parameters}] \leavevmode\begin{itemize}
\item {} 
\textbf{\texttt{data}} -- Matrix of data, where each column is a separate sample.

\item {} 
\textbf{\texttt{labels}} -- List of labels, each corresponding to a column of data.

\end{itemize}

\end{description}\end{quote}

\end{fulllineitems}


\end{fulllineitems}



\subsubsection{Module contents}
\label{eigenfish.classify:module-eigenfish.classify}\label{eigenfish.classify:module-contents}\index{eigenfish.classify (module)}

\subsection{eigenfish.process package}
\label{eigenfish.process:eigenfish-process-package}\label{eigenfish.process::doc}

\subsubsection{Submodules}
\label{eigenfish.process:submodules}

\subsubsection{eigenfish.process.math module}
\label{eigenfish.process:eigenfish-process-math-module}\label{eigenfish.process:module-eigenfish.process.math}\index{eigenfish.process.math (module)}\index{fft2\_series() (in module eigenfish.process.math)}

\begin{fulllineitems}
\phantomsection\label{eigenfish.process:eigenfish.process.math.fft2_series}\pysiglinewithargsret{\sphinxcode{eigenfish.process.math.}\sphinxbfcode{fft2\_series}}{\emph{img\_mat}, \emph{shape}}{}
For each column in img\_mat, img\_mat{[}:, i{]} the fft2 modes are extracted and
placed into the corresponding column of the returned matrix.
\begin{quote}\begin{description}
\item[{Parameters}] \leavevmode\begin{itemize}
\item {} 
\textbf{\texttt{img\_mat}} -- Matrix to process.

\item {} 
\textbf{\texttt{shape}} -- Original (width, height) of each column of img\_mat.

\end{itemize}

\item[{Returns}] \leavevmode
New numpy.ndarray matrix, where return{[}:, i{]} is the fft2 modes of
img\_mat{[}:, i{]}.

\end{description}\end{quote}

\end{fulllineitems}

\index{rpca() (in module eigenfish.process.math)}

\begin{fulllineitems}
\phantomsection\label{eigenfish.process:eigenfish.process.math.rpca}\pysiglinewithargsret{\sphinxcode{eigenfish.process.math.}\sphinxbfcode{rpca}}{\emph{image\_mat}}{}
Performs Robust Principle Component Analysis on image\_mat.
\begin{quote}\begin{description}
\item[{Returns}] \leavevmode
Low-rank, sparse parts of image\_mat

\end{description}\end{quote}

\end{fulllineitems}



\subsubsection{eigenfish.process.process module}
\label{eigenfish.process:module-eigenfish.process.process}\label{eigenfish.process:eigenfish-process-process-module}\index{eigenfish.process.process (module)}\index{Processor (class in eigenfish.process.process)}

\begin{fulllineitems}
\phantomsection\label{eigenfish.process:eigenfish.process.process.Processor}\pysigline{\sphinxstrong{class }\sphinxcode{eigenfish.process.process.}\sphinxbfcode{Processor}}
Bases: \sphinxcode{object}
\index{process() (eigenfish.process.process.Processor method)}

\begin{fulllineitems}
\phantomsection\label{eigenfish.process:eigenfish.process.process.Processor.process}\pysiglinewithargsret{\sphinxbfcode{process}}{\emph{img\_mat}, \emph{shape}}{}
Process img\_mat to prepare it for training/classification.
\begin{quote}\begin{description}
\item[{Parameters}] \leavevmode\begin{itemize}
\item {} 
\textbf{\texttt{img\_mat}} -- Matrix with each column a flattened image.

\item {} 
\textbf{\texttt{shape}} -- Original (width, height) of each image.

\end{itemize}

\end{description}\end{quote}

\end{fulllineitems}


\end{fulllineitems}



\subsubsection{Module contents}
\label{eigenfish.process:module-eigenfish.process}\label{eigenfish.process:module-contents}\index{eigenfish.process (module)}

\section{Submodules}
\label{eigenfish:submodules}

\section{eigenfish.eigenfish module}
\label{eigenfish:eigenfish-eigenfish-module}\label{eigenfish:module-eigenfish.eigenfish}\index{eigenfish.eigenfish (module)}\index{Eigenfish (class in eigenfish.eigenfish)}

\begin{fulllineitems}
\phantomsection\label{eigenfish:eigenfish.eigenfish.Eigenfish}\pysiglinewithargsret{\sphinxstrong{class }\sphinxcode{eigenfish.eigenfish.}\sphinxbfcode{Eigenfish}}{\emph{shape}, \emph{training\_file=None}, \emph{processor=None}, \emph{classifier=None}}{}
Bases: \sphinxcode{object}
\index{classify() (eigenfish.eigenfish.Eigenfish method)}

\begin{fulllineitems}
\phantomsection\label{eigenfish:eigenfish.eigenfish.Eigenfish.classify}\pysiglinewithargsret{\sphinxbfcode{classify}}{\emph{img\_mat}}{}
Classify img\_mat based on current training.
\begin{quote}\begin{description}
\item[{Parameters}] \leavevmode
\textbf{\texttt{img\_mat}} -- Column-wise matrix of flattened images.

\item[{Returns}] \leavevmode
List of labels, one for each column of img\_mat.

\end{description}\end{quote}

\end{fulllineitems}

\index{cross\_validate() (eigenfish.eigenfish.Eigenfish method)}

\begin{fulllineitems}
\phantomsection\label{eigenfish:eigenfish.eigenfish.Eigenfish.cross_validate}\pysiglinewithargsret{\sphinxbfcode{cross\_validate}}{\emph{img\_mat}, \emph{label\_arr}}{}
Cross-validates the trained model. Img\_mat will be run through the
classifier, and each predicted label of img\_mat{[}:, i{]} compared with
label\_arr{[}i{]}. The percent same is returned.
\begin{quote}\begin{description}
\item[{Parameters}] \leavevmode\begin{itemize}
\item {} 
\textbf{\texttt{img\_mat}} -- Column-wise matrix of flattened images.

\item {} 
\textbf{\texttt{label\_arr}} -- List of labels, where label\_arr{[}i{]} corresponds to
img\_mat{[}:, i{]}.

\end{itemize}

\item[{Returns}] \leavevmode
Percent of labels that are the same.

\end{description}\end{quote}

\end{fulllineitems}

\index{load() (eigenfish.eigenfish.Eigenfish method)}

\begin{fulllineitems}
\phantomsection\label{eigenfish:eigenfish.eigenfish.Eigenfish.load}\pysiglinewithargsret{\sphinxbfcode{load}}{\emph{filename}}{}
Loads saved training data and overwrites current model. Use only on data
you have previously saved, and make sure to use the same processor and
classifier.
\begin{quote}\begin{description}
\item[{Parameters}] \leavevmode
\textbf{\texttt{filename}} -- File to load into classifier.

\end{description}\end{quote}

\end{fulllineitems}

\index{save() (eigenfish.eigenfish.Eigenfish method)}

\begin{fulllineitems}
\phantomsection\label{eigenfish:eigenfish.eigenfish.Eigenfish.save}\pysiglinewithargsret{\sphinxbfcode{save}}{\emph{filename}}{}
Saves currently trained model to filename.
\begin{quote}\begin{description}
\item[{Parameters}] \leavevmode
\textbf{\texttt{filename}} -- File to save from classifier.

\end{description}\end{quote}

\end{fulllineitems}

\index{train() (eigenfish.eigenfish.Eigenfish method)}

\begin{fulllineitems}
\phantomsection\label{eigenfish:eigenfish.eigenfish.Eigenfish.train}\pysiglinewithargsret{\sphinxbfcode{train}}{\emph{img\_mat}, \emph{label\_arr}}{}
Add to current model's training.
\begin{quote}\begin{description}
\item[{Parameters}] \leavevmode\begin{itemize}
\item {} 
\textbf{\texttt{img\_mat}} -- Column-wise matrix of flattened images.

\item {} 
\textbf{\texttt{label\_arr}} -- List of labels, where label\_arr{[}i{]} corresponds to
img\_mat{[}:, i{]}.

\end{itemize}

\end{description}\end{quote}

\end{fulllineitems}


\end{fulllineitems}



\section{eigenfish.util module}
\label{eigenfish:eigenfish-util-module}\label{eigenfish:module-eigenfish.util}\index{eigenfish.util (module)}\index{load\_img\_mat() (in module eigenfish.util)}

\begin{fulllineitems}
\phantomsection\label{eigenfish:eigenfish.util.load_img_mat}\pysiglinewithargsret{\sphinxcode{eigenfish.util.}\sphinxbfcode{load\_img\_mat}}{\emph{files}}{}
Loads all files as images in black and white, flattens them and returns them
as a numpy.ndarray.
\begin{quote}\begin{description}
\item[{Parameters}] \leavevmode
\textbf{\texttt{files}} -- List of image files to load. All should be of the same
resolution.

\item[{Returns}] \leavevmode
Numpy.ndarray matrix with each column a flattened images.

\end{description}\end{quote}

\end{fulllineitems}



\section{Module contents}
\label{eigenfish:module-contents}\label{eigenfish:module-eigenfish}\index{eigenfish (module)}

\chapter{Indices and tables}
\label{index:indices-and-tables}\begin{itemize}
\item {} 
\DUrole{xref,std,std-ref}{genindex}

\item {} 
\DUrole{xref,std,std-ref}{modindex}

\item {} 
\DUrole{xref,std,std-ref}{search}

\end{itemize}


\renewcommand{\indexname}{Python Module Index}
\begin{theindex}
\def\bigletter#1{{\Large\sffamily#1}\nopagebreak\vspace{1mm}}
\bigletter{e}
\item {\texttt{eigenfish}}, \pageref{eigenfish:module-eigenfish}
\item {\texttt{eigenfish.classify}}, \pageref{eigenfish.classify:module-eigenfish.classify}
\item {\texttt{eigenfish.classify.classify}}, \pageref{eigenfish.classify:module-eigenfish.classify.classify}
\item {\texttt{eigenfish.eigenfish}}, \pageref{eigenfish:module-eigenfish.eigenfish}
\item {\texttt{eigenfish.process}}, \pageref{eigenfish.process:module-eigenfish.process}
\item {\texttt{eigenfish.process.math}}, \pageref{eigenfish.process:module-eigenfish.process.math}
\item {\texttt{eigenfish.process.process}}, \pageref{eigenfish.process:module-eigenfish.process.process}
\item {\texttt{eigenfish.util}}, \pageref{eigenfish:module-eigenfish.util}
\end{theindex}

\renewcommand{\indexname}{Index}
\printindex
\end{document}
